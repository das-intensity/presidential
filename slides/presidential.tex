
\documentclass[xcolor=usenames]{beamer} % aspectratio=169

\usepackage{fancyvrb}
\usepackage{graphics}
\usepackage[utf8]{inputenc}
\usepackage{amsmath}
%\usepackage{lmodern}
\usepackage{amssymb}
\usepackage{MnSymbol}
\usepackage[normalem]{ulem} % for strikeout
%\usepackage{verbatim} % TODO uncomment later
%\usepackage{tikz}
\usepackage{textcomp} % copy-pastable chars
\usepackage{hyperref}
\hypersetup{
	pdfauthor={Nicholas Dahm (Dr. Nick)},
	pdfsubject={Computer Vision},
	pdftitle={Hacking the Visual World with OpenCV},
	pdfkeywords={OpenCV, Python}
}

\mode<presentation>
{
	%\usetheme{Warsaw}
	\usetheme{Madrid}
	%\usetheme{Torino}
	\setbeamercovered{transparent}

	\usefonttheme{serif}

	% COLOURS
	%\usecolortheme{spruce}
	%\setbeamercolor{local structure}{fg=??} % Make everything inherit theme colours

	\usecolortheme{beaver} % needs different bullet colours
	\setbeamercolor{local structure}{fg=darkred} % Make everything inherit theme colours

	%\setbeamercolor{item projected}{bg=darkred}
	%\setbeamertemplate{enumerate items}[default]
	%\setbeamercolor{block title}{fg=darkred}

	\setbeamertemplate{navigation symbols}{}


	\setbeamertemplate{itemize item}{$\blacktriangleright$}
	%\setbeamertemplate{itemize subitem}{$\triangleright$}
	\setbeamertemplate{itemize subitem}{-}
}

\AtBeginSection[]
{
	\begin{frame}<beamer>{}
		\tableofcontents[currentsection,currentsubsection]
	\end{frame}
}

\AtBeginSubsection[]
{
	\begin{frame}<beamer>{}
		\tableofcontents[currentsection,currentsubsection]
	\end{frame}
}


\newcommand{\bi}{\begin{itemize}}
\newcommand{\ei}{\end{itemize}}

\newcommand{\be}{\begin{enumerate}}
\newcommand{\ee}{\end{enumerate}}

\newcommand{\degrees}{^\circ}

\newcommand{\bc}[1][0.5]{\begin{column}{#1\textwidth}}
\newcommand{\ec}{\end{column}}

\newcommand{\bci}[1][0.5]{\begin{column}{#1\textwidth} \begin{itemize}}
\newcommand{\eci}{\end{itemize}\end{column}}

% This allows putting images arbitrarily on slides
\def\Put(#1,#2)#3{\leavevmode\makebox(0,0){\put(#1,#2){#3}}}


\usepackage{listings}
\usepackage{color}

\definecolor{dkgreen}{rgb}{0,0.6,0}
\definecolor{gray}{rgb}{0.5,0.5,0.5}
\definecolor{mauve}{rgb}{0.58,0,0.82}

\lstset{frame=tb,
	language=python,
	aboveskip=3mm,
	belowskip=3mm,
	showstringspaces=false,
	columns=flexible,
	basicstyle={\small\ttfamily},
	numbers=none,
	numberstyle=\tiny\color{gray},
	keywordstyle=\color{blue},
	commentstyle=\color{dkgreen},
	stringstyle=\color{mauve},
	breaklines=true,
	breakatwhitespace=true,
	tabsize=4,
	upquote=true
}





\title[Hacking the Visual World]{Hacking the Visual World with OpenCV}
\subtitle{Computer Vision in Python}
\date{January 25, 2017}
\author[Dr. Nick]{Nicholas Dahm / Dr. Nick / Aussie Nick}
\institute[TrademarkVision]{
	\Large
	TrademarkVision\\\vspace{4mm}
	\normalsize
	\textit{Code \& slides available at:}\\\url{https://github.com/das-intensity/presidential}
}
\subject{Computer Vision} % PDF properties only


%%% IMAGE SOURCES IN THIS PRESENTATION %%%
% NOTE: Most images in this presentation were listed as available for non-commercial reuse by google or via public domain image sites. Please directly confirm the copyright status of them before reuse.
% EXCEPTIONS:
%-The tmv-results.png image was taken via screenshot of the trademark.vision system. The logos within it are of course copyright protected, and as such, this image should not be reused elsewhere.
%-The tmv-logo.png is copyright protected, but permission was obtained to include it in this presentation. Reuse prohibited.

% face-tracking.png <- https://upload.wikimedia.org/wikipedia/commons/4/4e/Visage_Technologies_Face_Tracking_and_Analysis.png
% nasa-robot.jpg <- https://c3.staticflickr.com/5/4083/5161877250_7bc25c9e19_b.jpg
% fpv-hud.png <- https://upload.wikimedia.org/wikipedia/commons/d/d2/FPV_OSD.png
% camera-transformer.jpg <- https://c1.staticflickr.com/2/1062/835831228_7e2bc6a207.jpg
% hyperspectral.png <- https://upload.wikimedia.org/wikipedia/en/7/77/Specim_aisaowl_outdoor.png
% python.svg <- https://pixabay.com/get/ea34b30a2ef51c3e955e4600e1444e9efe76e6dd1cb2134290f5c1/snake-312561.svg?attachment
% opencv.svg <- https://upload.wikimedia.org/wikipedia/commons/3/32/OpenCV_Logo_with_text_svg_version.svg?download
% numpy.png <- https://upload.wikimedia.org/wikipedia/en/1/1b/NumPy_logo.png?download
% mag.svg <- https://upload.wikimedia.org/wikipedia/commons/3/3a/Magnifying_glass_01.svg
% hsv.png <- https://upload.wikimedia.org/wikipedia/commons/thumb/0/0d/HSV_color_solid_cylinder_alpha_lowgamma.png/320px-HSV_color_solid_cylinder_alpha_lowgamma.png
% The *-hog.jpg images were created with the hog visualisation tool here: http://www.cs.cornell.edu/courses/cs4670/2012fa/projects/p5/index.html


\begin{document}
%\part<presentation>{Front}
\begin{frame}
	\titlepage
\end{frame}


\begin{frame}
	\tableofcontents
\end{frame}


\section{What is Computer Vision?}
\begin{frame}{What is Computer Vision?}
	\Put(0,100){\includegraphics[height=35mm]{images/fpv-hud}}% TOP LEFT
	\Put(0,-240){\includegraphics[height=35mm]{images/hyperspectral}}% BOTTOM LEFT
	%
	\Put(180,150){\includegraphics[height=35mm]{images/nasa-robot}}% TOP RIGHT
	\Put(240,-240){\includegraphics[height=35mm]{images/face-tracking}}% BOTTOM RIGHT
	%
	\Put(100,0){\includegraphics[height=35mm]{images/camera-transformer}}% MIDDLE
\end{frame}


\begin{frame}{What is the data?}
	\Put(20,100){\includegraphics[height=30mm]{images/mantis-shrimp}}%
	\Put(0,-200){\includegraphics[height=30mm]{images/mantis-shrimp-grey}}%
	\Put(210,20){\includegraphics[height=30mm]{images/mantis-shrimp-blue}}%
	\Put(180,-150){\includegraphics[height=30mm]{images/mantis-shrimp-green}}%
	\Put(150,-240){\includegraphics[height=30mm]{images/mantis-shrimp-red}}%
\end{frame}


\begin{frame}{How to Computer Vision a thing?}
	\bi
		\item \textbf{Object detection} - where are the faces in this image?
		\item \textbf{Object recognition} - is that face John Smith?
		\item \textbf{Object classification} - which animal is this?
		\item \textbf{Image matching} - how similar are these images?
		\item \textbf{Image retrieval} - what images look similar to this?
		\item \textbf{Object tracking} - watch where that car goes
		\item \textbf{Scene reconstruction} - create a 3D model from this video
		\item \textbf{Image restoration} - remove the wrinkle from this scanned photo
	\ei
\end{frame}


\begin{frame}{TrademarkVision - image retrieval}
	%\Put(0,100){\includegraphics[height=35mm]{images/tmv-results}}% TOP LEFT
	\centering
	\includegraphics[height=0.8\textheight]{images/tmv-results}
\end{frame}


\begin{frame}{Image features - HOG}
	\Put(0,100){\includegraphics[height=35mm]{images/tmv-logo}}% TOP LEFT
	\Put(160,100){\Huge $\rightarrow$}% TOP LEFT
	\Put(230,100){\includegraphics[height=33mm]{images/tmv-logo-hog}}% TOP RIGHT
	%
	\Put(20,-240){\includegraphics[height=35mm]{images/mag}}% BOTTOM LEFT
	\Put(160,-140){\Huge $\rightarrow$}% TOP LEFT
	\Put(240,-240){\includegraphics[height=35mm]{images/mag-hog}}% BOTTOM RIGHT
	%Talk about tracking, segmentation, image matching. Give some examples, e.g. what we do at TMV
\end{frame}


\begin{frame}{Comparing image features}
	\Put(0,100){\includegraphics[height=35mm]{images/tmv-logo-hog}}% TOP LEFT
	\Put(140,100){\Huge $\rightarrow$}% TOP LEFT
	\Put(180,100){$(0.20, 0.04, 0.07, 0.12, \ldots, 0.11)$}% TOP RIGHT
	%
	\Put(150,0){Diff $= (0.07, 0.14, 0.06, 0.08, \ldots, 0.01)$}%
	\Put(157,-40){$L_1 = 0.52$}%
	%
	\Put(10,-240){\includegraphics[height=35mm]{images/mag-hog}}% BOTTOM LEFT
	\Put(140,-140){\Huge $\rightarrow$}% TOP LEFT
	\Put(180,-140){$(0.13, 0.18, 0.01, 0.04, \ldots, 0.12)$}% TOP RIGHT
	%Talk about tracking, segmentation, image matching. Give some examples, e.g. what we do at TMV
\end{frame}


\begin{frame}{Computer Vision in Python}
	What Computer Vision tools are available for us in Python?
	\bi
		\item \textbf{OpenCV}
		\bi
			\item Large library of computer vision functions
			\item Many algorithms for all types of tasks
			\item Python wrapper for underlying C++ library
			\item Stores images as NumPy arrays for easy manipulation
		\ei
		\item \textbf{NumPy}
		\bi
			\item Extensive matrix manipulation library
			\item Incredibly fast when using matrix operations
		\ei
		\item \textbf{PIL}
		\bi
			\item Python Imaging Library (or newer Pillow fork)
			\item Various image operations, including some not in OpenCV
			\item Quick translations to/from OpenCV's NumPy format
		\ei
	\ei
\end{frame}


\section{Example: The Presidential Look}
\begin{frame}{Prerequisites}
	\bi
		\item Python (2.7 recommended)
		\item OpenCV (3 recommended)
		\item NumPy
		\item PIL (optional)
	\ei
	\Put(10,-220){\includegraphics[height=40mm]{images/python}}%
	\Put(170,50){\includegraphics[height=50mm]{images/opencv}}%
	%\Put(180,-200){\fbox{\bf \Huge NumPy}}%
	\Put(180,-220){\includegraphics[height=15mm]{images/numpy}}%
	\vspace{40mm}
\end{frame}


\begin{frame}[fragile]{A Wild Squirrel Appears!}
	\Put(220,-220){\includegraphics[height=35mm]{images/squirrel}}%
	\vspace{-2mm}
	\begin{lstlisting}
import cv2
import numpy as np

# Download goo.gl/nXaoEf as squirrel.png

sq = cv2.imread('squirrel.png')
print sq.shape # (140, 160, 3)
cv2.imshow('squirrel', sq)

sq_gray = cv2.cvtColor(sq, cv2.COLOR_BGR2GRAY)
print sq_gray.shape # (140, 160)
cv2.imshow('squirrel gray', sq_gray)
cv2.waitKey()
	\end{lstlisting}
	\vspace{-1mm}
	\bi
		\item You used computer vision... it's super effective!
	\ei
\end{frame}


\begin{frame}{OpenCV Squirrel Windows}
	\Put(40,0){\includegraphics[height=45mm]{images/step-squirrels}}%
\end{frame}


% TODO remove this slide?
\begin{frame}[fragile]{PIL~\textless-\textgreater~OpenCV}
	\begin{lstlisting}
from PIL import Image
pilsq = Image.open('squirrel.png')

# PIL -> cv2
sq2 = np.asarray(pilsq)
sq2 = cv2.cvtColor(sq2, cv2.COLOR_RGB2BGR)
# cv2 -> PIL
pilsq2 = cv2.cvtColor(sq2, cv2.COLOR_BGR2RGB)
pilsq2 = Image.fromarray(pilsq2)
	\end{lstlisting}
\end{frame}


\begin{frame}[fragile]{I wanna see myself!}
	\begin{lstlisting}
print 'starting webcam...'
cam = cv2.VideoCapture(0)
ret, frame = cam.read()
assert ret # fails if couldn't read from webcam
print 'webcam res:', frame.shape

while True:
    ret, frame = cam.read()

    cv2.imshow('cam', frame)
    key = cv2.waitKey(1)
    if key != -1:
        break
	\end{lstlisting}
\end{frame}


\begin{frame}{Hi everybody!}
	\Put(90,0){\includegraphics[height=45mm]{images/step-webcam}}%
\end{frame}


\begin{frame}[fragile]{I still want the squirrel though!}
	\begin{lstlisting}
sq_gray_mask = sq_gray > 0
sq_mask = np.array(
		[[[v, v, v] for v in row] for row in sq_gray_mask]
		)

while True:
    ret, frame = cam.read()

    # overlay the squirrel
    np.copyto(
			frame[-sq.shape[0]:, 0:sq.shape[1], :],
			sq,
			where=sq_mask
			)

    cv2.imshow('cam', frame)
	\end{lstlisting}
\end{frame}


\begin{frame}{A ``Squirrel Productions'' film}
	\Put(90,0){\includegraphics[height=45mm]{images/step-squirrel-productions}}%
\end{frame}


\begin{frame}[fragile]{Detecting faces}
	\begin{lstlisting}
cas = cv2.CascadeClassifier(
	'/usr/share/opencv/haarcascades/'
	+ 'haarcascade_frontalface_default.xml'
	)
	\end{lstlisting}
	\begin{lstlisting}
	### inside while loop, after cam.read() ###
    frame_gray = cv2.cvtColor(frame, cv2.COLOR_BGR2GRAY)
    faces = cas.detectMultiScale(
            frame_gray,
            scaleFactor=1.1,
            minNeighbors=5,
            minSize=(30, 30),
            flags = cv2.CASCADE_SCALE_IMAGE
            # OpenCV 2.4 -> cv2.cv.CV_HAAR_SCALE_IMAGE
            )
	# faces is list of (x, y, width, height)
	\end{lstlisting}
\end{frame}


\begin{frame}[fragile]{Drawing faces}
	\begin{lstlisting}
    for (x, y, w, h) in faces:
        cv2.rectangle(frame, (x, y), (x+w, y+h), (0, 255, 0), 2)

		# Make the box a bit bigger (see why later)
		#- first save originals
        x0 = x; y0 = y; w0 = w; h0 = h

        h += int(h * 0.3)
        x -= int(w * 0.1)
        w += int(w * 0.2)
        if y < 0 or y+h > frame.shape[0]: continue
        if x < 0 or x+w > frame.shape[1]: continue

        cv2.rectangle(frame, (x, y), (x+w, y+h), (0, 255, 255), 2)
	\end{lstlisting}
\end{frame}


\begin{frame}{FaceBox.. the latest social media craze}
	\Put(90,0){\includegraphics[height=45mm]{images/step-face-detection}}%
\end{frame}


\begin{frame}[fragile]{Find the non-presidential skin...}
	\begin{lstlisting}
		### inside faces loop ###
        # Find the skin pixels
        frame_hsv = cv2.cvtColor(frame, cv2.COLOR_BGR2HSV)
        face_hsv = frame_hsv[y:y+h, x:x+w, :]

        hsv_min = np.array([0, 0, 40])
        hsv_max = np.array([200, 120, 220])
        face_inrange = cv2.inRange(face_hsv, hsv_min, hsv_max)
        face_inrange = np.array([[[v, v, v] for v in row] for row in face_inrange], dtype=bool)
        #print 'face_inrange: %s - %s' % (face_inrange.shape, face_inrange.dtype)
	\end{lstlisting}
\end{frame}

\begin{frame}[fragile]{Upgrade it to a more presidential colour}
	\begin{lstlisting}
		### inside faces loop ###
		# Set a more presidential color
        pres_skin = [0.2, 0.65, 1.7]
        face_bgr = frame[y:y+h, x:x+w, :] * pres_skin
        face_bgr = np.minimum(face_bgr, 255)
        face_bgr = np.array(face_bgr, dtype='uint8')
        #print 'face_bgr: %s - %s' % (face_bgr.shape, face_bgr.dtype)
        np.copyto(frame[y:y+h, x:x+w, :], face_bgr, where=face_inrange)
	\end{lstlisting}
\end{frame}


\begin{frame}{Orange is the presidential colour}
	\Put(90,0){\includegraphics[height=45mm]{images/step-presidential-skin}}%
\end{frame}


\begin{frame}[fragile]{Let get some presidential hair!}
	\begin{lstlisting}
		### inside faces loop ###
        # Obtain some presidential hair
        hair = sq[30:,0:70]
        hair = np.rot90(hair, 3)
        hair_mask = sq_mask[30:,0:70]
        hair_mask = np.rot90(hair_mask, 3)

        hair_scale = w0 / float(hair.shape[1])

        hair = cv2.resize(hair, None, fx=hair_scale, fy=hair_scale)
	\end{lstlisting}
\end{frame}

\begin{frame}[fragile]{Wear the presidential hair!}
	\begin{lstlisting}
		### inside faces loop ###
		# Wear the presidential hair!
        hair_mask = np.array(hair_mask, dtype='uint8')
        hair_mask = cv2.resize(hair_mask, None, fx=hair_scale, fy=hair_scale)
        hair_mask = np.array(hair_mask, dtype=bool)

        hair = np.array(np.minimum((hair * 0.7) + 100, 255), dtype='uint8') # Optional

        if y0 - hair.shape[0] < 0: continue
        np.copyto(frame[y0-hair.shape[0]:y0, x0:x0+hair.shape[1], :], hair, where=hair_mask)
	\end{lstlisting}
\end{frame}


\begin{frame}{The most presidential look, period!}
	\Put(90,0){\includegraphics[height=45mm]{images/step-presidential-hair}}%
\end{frame}


\begin{frame}{Thanks for listening!}
	\centering
	\textbf{Never let the visual reality get in your way again!} \\\vspace{4mm}
	\includegraphics[height=45mm]{images/computer-eye}\\
	\href{https://github.com/das-intensity}{github.com/das-intensity}\\
	\href{https://www.linkedin.com/in/nicholasdahm}{linkedin.com/in/nicholasdahm}\\
	\href{https://meetup.com/members/195619601}{meetup.com/members/195619601}
\end{frame}


\begin{frame}{Want to Computer some Visions?}
	\Put(0,20){\includegraphics[width=\textwidth]{images/logo_dark_transparent}}%
	\vspace{12mm}
	\bi
		\item \textbf{Computer Vision Interns} \vspace{2mm}
		\item \textbf{Experienced Python Django Developer} \vspace{2mm}
		\item \textbf{DevOps \& Data Engineer} \vspace{4mm}
	\ei
	\hspace{6mm} \textit{Contact:} \href{mailto:ask@trademark.vision}{ask@trademark.vision} \\ \vspace{4mm}
	\hspace{6mm} \textit{Info:} \href{https://trademark.vision/about/trademarkvision-careers}{trademark.vision/about/trademarkvision-careers}

\end{frame}

\end{document}

